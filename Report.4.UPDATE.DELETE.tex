\documentclass[]{article}
\begin{document}
\begin{enumerate}
\item Steps \& Explanation
\renewcommand{\labelenumii}{\Roman{enumii}}
\begin{enumerate}
\item Insert a new record into the department table with dept\_no='d010' and dept\_name='Research and 							Development'
We need this new row in the database to have a department called 'Research and Development' so as to be able to
update employees in the Development and Research department to our newly created department.
\\
\item Insert a new record into the department table with dept\_no='d011' and dept\_name='Marketing and Sales'
We need this new row in the database to have a department called 'Marketing and Sales' so as to be able to
update employees in the Marketing and Sales department to our newly created department.
\\
\item We need to identify the tables that uses the dept\_no and update the dept\_no. The tables in this case are 				dept\_emp and dept\_manager. 
\\
\item We need to update the dept\_no 'd005' and 'd008' to 'd010' like wise we need to update dept\_no 				 			'd001' and 'd007' to 'd011'. These operation should be performed on the dept\_emp and dept\_manager tables.
\\
\item When updating we need to use the 'IGNORE' keyword after the update. The reason to use this keyword is because if 			there is a conflict with the primary key you won't be able to replace the department. Using this keyword helps 					delete the record and replaces it with the new record.
\end{enumerate}
\item SQL commands and output
\begin{itemize}
\item INSERTING INTO DEPARTMENTS
\\\\
INSERT INTO departments(dept\_no,dept\_name) VALUES ('d010','Research and Development');
\\
Query OK, 1 row affected (0.08 sec)
\\\\
INSERT INTO departments(dept\_no,dept\_name) VALUES ('d011','Marketing and Sales');
\\
Query OK, 1 row affected (0.06 sec)
\\\\
\item UPDATING DEPARTMENT IN DEPT\_EMP
\\\\
UPDATE IGNORE dept\_emp SET dept\_no='d010' WHERE dept\_no='d005';
\\
Query OK, 85707 rows affected (21.11 sec)
\\
Rows matched: 85707  Changed: 85707  Warnings: 0
\\\\
UPDATE IGNORE dept\_emp SET dept\_no='d010' WHERE dept\_no='d008';
\\
Query OK, 15892 rows affected, 5234 warnings (6.35 sec)
\\
Rows matched: 21126  Changed: 15892  Warnings: 5234
\\\\
UPDATE IGNORE dept\_emp SET dept\_no='d011' WHERE dept\_no='d001' OR dept\_no='d007';
\\
Query OK, 68804 rows affected, 3652 warnings (12.34 sec)
\\
Rows matched: 72456  Changed: 68804  Warnings: 3652
\\
\item UPDATING DEPARTMENT IN DEPT\_MANAGER
\\\\
UPDATE IGNORE dept\_manager SET dept\_no='d010' WHERE dept\_no='d005' OR dept\_no='d008';
\\
Query OK, 4 rows affected (0.16 sec)
\\
Rows matched: 4  Changed: 4  Warnings: 0
\\\\
UPDATE IGNORE dept\_manager SET dept\_no='d011' WHERE dept\_no='d001' OR dept\_no='d007';
\\
Query OK, 4 rows affected (0.08 sec)
\\
Rows matched: 4  Changed: 4  Warnings: 0
\\	
\item DELETING FROM DEPARTMENT
DELETE FROM departments WHERE dept\_no='d001' OR dept\_no='d005' OR dept\_no='d007' OR dept\_no='d008';
\\
Query OK, 4 rows affected (0.38 sec)
\end{itemize}
\end{enumerate}
\end{document}
